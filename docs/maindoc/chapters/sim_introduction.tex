\chapter{Simulation}

\section{Introduction}
How we said in the previous chapter we built the model inside the framework \textbf{OMNeT++ v5}. The definition of network \textit{(CellularNetwork.ned)} and their components can be found in the directory \textit{RRCellNet/src}. We decided to analyze it when 10 \texttt{Mobile Stations} are connected to the \texttt{Antenna}.

We recall that \texttt{Web Servers} are sources of packets addressed to \texttt{Mobile Stations}. In order to have true random and independent arrivals to \texttt{Antenna} we need a 10 RNG, one for each \texttt{Web Server}. We also need RNG to generate packets with random size so others 10 RNG are neeeded. We need also a RNG for each \texttt{Mobile Station} in order to generate random CQIs at each timeslot. Overall the model requires 30 RNGs each of them initialized with a different seed in order to have independent pseudo-random variables. The seed-set is changed at every repetition to have different independent experiments. At the end of all repetions the results are aggregated by computing the mean and 95\% confidence interval. 

Let's consider for example the mean throughput for a rate \(\lambda^{*}\). By running each repetition we get the values \(X_{1},X_{2},\ldots X_{10}\). \(X_{i}\) is a random variable which represents the mean throughput for the repetition \(i\). By the CLT theorem we can say that \(X_{i}\) is a normal RV since it is obtained by summing up a huge number of \textit{slotted throughput}. We can estimate the mean \(\bar{X}\) and a 95\% confidence interval by using the \textit{Student's t distribution} because \(X_{i}\) are normal RV. 

\begin{align}
	\bar{X} &= \frac{X_{1}+X_{2}+\ldots+X_{10}}{10} \qquad S^{2} = \frac{1}{9}\sum_{i=1}^{10}(X_{i} - \bar{X})^{2} \\
	CI_{0.95} &= \left[\bar{X} - \frac{S}{\sqrt10}t_{0.025,9}, \: \bar{X} + \frac{S}{\sqrt10}t_{0.025,9}\right]
\end{align}

Similar consideration can be done for others quanties wich we will analyze during simulations. Once we have computed the mean \(\bar{X}\) and its confidence interval we can do a box plot for that quantity at varying workload \(\lambda\) as required by specifications. Data are exported from simulation to csv files through a bash script \textit{exportdata.sh} and plots are done through an R script \textit{analyze\_csv.r}. These script are both included in the directory \textit{RRCellNet/simulations}.

All parameters for simulations are summarized here and can be found in the file \textit{RRCellNet/simulations/omnetpp.ini}. We will use them in the following chapters unless otherwise specified.
\begin{itemize}
\item \textbf{Number of resource block}, \(\#RB = 25\)
\item \textbf{Number of users}, \(n = 10\)
\item \textbf{Number of RNG}, \(\#RNG = 30\)
\item \textbf{Max RB size}, \(RBsize\textsubscript{max} = 93\)
\item \textbf{Max packet size}, \(packetsize\textsubscript{max} = 75\)
\item \textbf{Timeslot period}, \(T\textsubscript{slot} = 1\textnormal{ms}\)
\item \textbf{Number of repetions}, \(\#REP = 10\)
\item \textbf{Simulation time}, \(ST = 60 \textnormal{s}\)
\item \textbf{Warmup period}, \(WP = 0.3 \textnormal{s}\)
\end{itemize}
