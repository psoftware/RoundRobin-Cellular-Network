\chapter{Simulation: Binomial Scenarios}
\section{Introduction}
This scenario models an enviroment where users have a fixed position in the space. There are some users which receive stronger signal from antenna, so they have an higher mean CQI, and others which are far from antenna and then they have smaller mean CQI. The requirements state that the distribution of CQIs must have a binomial distribution. To satisfy these, at each timeslot \(t_{j}\), every \texttt{Mobile Station} \(i\) generates a RV \(X_{i,t_{j}} \sim Bin(n-1,p_{i})\), and then \(CQI_{i,t_{j}} = X + 1\). By using this trick \(CQI_{i,t_{j}} \in \{1,15\}\) and it has a binomial distribution. In order to have different mean we have chose this values for parameters \(p_{i}\).
\begin{lstlisting}[caption={omnet.ini - p parameters}]
	CellularNetwork.users[0].cqi_binomial_p = 0.13
	CellularNetwork.users[1].cqi_binomial_p = 0.22
	CellularNetwork.users[2].cqi_binomial_p = 0.31
	CellularNetwork.users[3].cqi_binomial_p = 0.40
	CellularNetwork.users[4].cqi_binomial_p = 0.49
	CellularNetwork.users[5].cqi_binomial_p = 0.58
	CellularNetwork.users[6].cqi_binomial_p = 0.67
	CellularNetwork.users[7].cqi_binomial_p = 0.76
	CellularNetwork.users[8].cqi_binomial_p = 0.85
	CellularNetwork.users[9].cqi_binomial_p = 0.94
\end{lstlisting}
As in the previous scenario we will analyze the performance about throughput and response time by using both scheduler. Before doing simulation we can do some conjectures. Note that in this scenario mean CQI are sensibly different so we expect that \(user[9]\), that has highest probability to generate high CQI, and it will have highest throughput. Morover we can suppose that it will take advantage of Best CQI policy annd will increase its throughput despite other users.